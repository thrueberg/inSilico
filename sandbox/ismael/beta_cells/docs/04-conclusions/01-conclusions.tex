The multicellular model framework presented in this work has been shown reliable to simulate individual and collective cell migration in 3D and its viability in different kinds of in vitro experimentation examples. Several phenomena are represented in the three integrated models as mechano-chemical cell to cell interaction or diffusion effects in the culture. Conclusions obtained by analyzing simulation data seem to match some of the reference consulted.\\
\\
Nevertheless, biological systems are highly complex and some processes need to be simplified to be represented in silico. Despite simplifications, different models can achieve a good approximation to specific tasks, consequently, an integration of several simple models is a feasible way to reach more flexible simulations of complicated systems.\\
\\\\

\textbf{Future Work}\\
\\
Modular nature of this work would allows the inclusion of more interesting biological processes in the future, for example, extended viability systems, multi-component diffusion, complex reactions and product synthesis. Existing models can be individually expanded in this framework and modify without altering the others, it is possible to add more models to simulate additional cell behavior, ECM properties or global experiment conditions.\\
\\
Firstly, cell proliferation would be an important addition to represent cultures that last enough to let cells to replicate. This addition with a improved viability system with other damage to the cell conditions would allow this system to simulate in vitro experimentation at higher time scales. Cell proliferation and viability are key factors to the correct development of tissues, but it is also important to be aware of the different kind of cells roles and properties. Therefore, a model that consider different type of cells, and maybe cell differentiation (for example from stem cells), could simulate highly complex processes and ultimately a fully functional tissue.\\
\\
Media depletion and multi-component diffusion can be useful to describe more dynamic environmental conditions that allow the simulation reach accurate results considering in vitro experimentation. Another interesting addition could be chemotaxis due to non-cellular component gradient in the culture, this would let study the influence of several compounds that attract or repel cells.\\
\\
Finally, mechanical interaction between cells could also grow in complexity and include viscoelastic effects due to cytoplasm and  mechanical deformation of the ECM. Cell-ECM interaction is considered an important factor to take in account by both mechanical and chemical point of view, so it inclusion can drive to even more precise in silico simulations. In the model used, cells are considered spherical, this is only valid to certain cases and could be improved considering more complex and dynamic cell geometries. 