Cell migration is a fundamental process in multicellular organism, it is central in tissue development, immune response, regeneration and angiogenesis among other biological  process (\cite{Rorth2009}). It is also important in many diseases, remarkably in cancer development and spreading (\cite{Polacheck2013}), it is known as a crucial factor in metastasis.\\
\\
This process can be observed both as individual and collective migration. Different kinds of cells can migrate through specific environment on their own way, adapted to their function in the organism. For example, lymphocytes has a high motility through human tissues to accomplish their immunological purpose (\cite{Rorth2009}), migrating as individuals. In the pancreas, cells migrate to form typical structure in the organ known as islets of Langerhans (\cite{Dahl1996}), in this case, as a collective or cluster of cells. An example of a cluster of cells is shown in Figure \ref{fig:introcluster}.\\
\\
There are some studies of migration in vivo but they are limited to the given conditions in the organism and the available tracking techniques. A  good example of it can be found in (\cite{Hoehn2002}), the experiments are based in monitoring by MRI modified stem cells producing GFP. Although, obtained results are important, these studies do not reveal the basis of the cell migration and the highly complex environment makes even harder to get specific data about individual cell behavior.\\

\begin{figure}[h]
\centering
\setlength\fboxsep{0pt}
\setlength\fboxrule{0.5pt}
\fbox{\includegraphics[width=200pt]{img/Intro-Cluster}}
\caption{Cell cluster image, electron microscopy. Hubrecht, http://www.hubrecht.eu}
\label{fig:introcluster}
\end{figure}


Additionally, development of new experimental techniques and materials allows to obtain more accurate results in vitro and replicate conditions of functional tissues and organs (\cite{Dunn2006}, \cite{Orlando2011}). Among other advantages, in vitro experimentation can be much more flexible at setting and controlling environmental cell conditions than in vivo experimentation, this fact is central to unravel the basis of different biological processes including individual and collective cell migration.\\
\\
Nowadays, there are several documented approaches to study cell migration. Most of investigations were mainly focused on 2D migration (\cite{Trepat2009}) until recent days but there is a substantial increase of new studies that aim cell behavior understanding in a 3D environment (\cite{DaRocha-Azevedo2013}).  Although results from 2D-approach are valuable, these are far from reaching a full explanation of cell activity in tissues that require three-dimensional structures to achieve real functionality.\\
\\
For example, it is well-known that oxygen and nutrient diffusion is a limiting factor when cells are grown in 3D in vitro cultures and not in 2D cultures. Therefore, low concentration of required compounds in the 3D environment can drive to situations of anomalous migration, abnormal cluster formation or straight to the cell death. In consequence, nutrients and oxygen diffusivity limits the culture size depending on material used as extra-cellular matrix (ECM) and it is necessary to take in account diffusion to fully represent cell behavior in a 3D culture.\\
\\